\chapter*{Conclusion générale}
\addcontentsline{toc}{chapter}{Conclusion générale}
\markboth{Conclusion générale}{}

\section*{Bilan du projet}

Le présent rapport a présenté la conception et le développement de \textbf{SkillSync}, une plateforme intelligente d'accompagnement de carrière basée sur l'intelligence artificielle. Ce projet de fin d'études a permis de mettre en pratique les connaissances acquises durant notre formation en génie logiciel, tout en explorant les technologies de pointe dans le domaine du NLP et du Machine Learning.

\subsection*{Objectifs atteints}

Les objectifs initiaux du projet ont été pleinement réalisés :

\begin{itemize}
    \item \textbf{Analyse intelligente de CV} : Le système utilise avec succès le modèle BERT NER pour extraire et catégoriser les compétences des CV, avec un score ATS multicritères explicable.
    
    \item \textbf{Génération automatique de portfolio} : La plateforme propose 5 templates professionnels personnalisables, permettant aux utilisateurs de créer un portfolio en quelques clics.
    
    \item \textbf{Matching sémantique} : L'intégration de Sentence-Transformers permet un matching CV-offres d'emploi basé sur la compréhension sémantique, dépassant la simple correspondance par mots-clés.
    
    \item \textbf{Guidance de carrière} : Le module de recommandations offre des parcours d'évolution personnalisés avec estimation du temps d'acquisition des compétences.
    
    \item \textbf{Architecture moderne} : L'application repose sur une stack technique robuste (FastAPI, React, PostgreSQL) avec une authentification JWT sécurisée.
\end{itemize}

\subsection*{Compétences développées}

Ce projet m'a permis de développer et renforcer de nombreuses compétences :

\begin{itemize}
    \item \textbf{Techniques} : NLP/NER avec BERT, embeddings vectoriels, développement full-stack, architecture REST API, authentification JWT
    
    \item \textbf{Méthodologiques} : Gestion de projet Scrum, planification par sprints, documentation technique
    
    \item \textbf{Transversales} : Résolution de problèmes complexes, autonomie, communication technique
\end{itemize}

\section*{Difficultés rencontrées}

Plusieurs défis ont été relevés durant le projet :

\begin{itemize}
    \item \textbf{Performance des modèles ML} : L'optimisation du temps de chargement et d'inférence du modèle BERT a nécessité la mise en place d'un système de cache.
    
    \item \textbf{Parsing multi-format} : La gestion des différents formats de CV (PDF, DOCX, scannés) a requis l'intégration de plusieurs bibliothèques et l'OCR.
    
    \item \textbf{Intégration d'APIs tierces} : La standardisation des réponses provenant de différentes APIs d'emploi a demandé un travail d'harmonisation.
\end{itemize}

\section*{Perspectives d'évolution}

SkillSync présente un potentiel d'évolution important :

\begin{itemize}
    \item \textbf{Court terme} :
    \begin{itemize}
        \item Ajout de nouveaux templates de portfolio
        \item Intégration d'APIs d'emploi supplémentaires
        \item Support multilingue (arabe, anglais)
        \item Application mobile (React Native)
    \end{itemize}
    
    \item \textbf{Moyen terme} :
    \begin{itemize}
        \item Fine-tuning d'un modèle NER spécialisé sur les CV
        \item Système de tracking de candidatures
        \item Intégration de LinkedIn pour import de profil
        \item Module de préparation aux entretiens avec IA
    \end{itemize}
    
    \item \textbf{Long terme} :
    \begin{itemize}
        \item Plateforme B2B pour les recruteurs
        \item Marketplace de formations
        \item Analyse prédictive des tendances du marché de l'emploi
        \item Personnalisation avancée via apprentissage continu
    \end{itemize}
\end{itemize}

\section*{Conclusion}

Le projet SkillSync démontre le potentiel de l'intelligence artificielle pour démocratiser l'accès à des outils d'aide à la recherche d'emploi. En combinant des technologies de pointe (NLP, embeddings, ML) avec une approche d'IA explicable, la plateforme répond à un besoin réel du marché tout en restant accessible au grand public.

Ce projet de fin d'études a été une expérience enrichissante qui m'a permis de concrétiser ma formation d'ingénieur en développant une solution complète, de la conception à la mise en production. Les compétences acquises et les défis relevés constituent une base solide pour ma future carrière dans le domaine du génie logiciel et de l'intelligence artificielle.

\vspace{1cm}

\begin{flushright}
\textit{« La technologie au service du développement professionnel »}
\end{flushright}
